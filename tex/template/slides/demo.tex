\providecommand{\notesroot}{../../..}
\documentclass[a4paper,10pt]{ctexbook}
\usepackage{xeCJK}
\usepackage{fontspec}
\usepackage{minted}
\usepackage[CJKbookmarks,colorlinks,linkcolor=red]{hyperref}
\usepackage{geometry}
\usepackage{amsmath}
\usepackage[format=hang,font=small,textfont=it]{caption}
\usepackage{float}
\usepackage{subfigure}
\usepackage[nottoc]{tocbibind}
\usepackage{bm}
\usepackage[table, x11names, dvipsnames]{xcolor}
\usepackage{color}
\usepackage{array, booktabs, boldline}
\usepackage{cellspace}
\usepackage{longtable}

\setmainfont{Times New Roman}
\setsansfont{Helvetica}
\setmonofont{Courier New}
\setCJKmainfont[BoldFont={SimHei},ItalicFont={SimHei}]{SimSun}
\setCJKsansfont{SimSun}
\setCJKmonofont{SimSun}

\setcounter{secnumdepth}{4}
\setcounter{tocdepth}{4}

\geometry{left=3.0cm,right=3.0cm,top=2.5cm,bottom=2.5cm}
\bibliographystyle{plain}

%%%%%%%%%%%%%%%%%%%%%%%%%%%%%%%%% minted setting %%%%%%%%%%%%%%%%%%%%%%%%%%%%%%%%%%%
\usemintedstyle{monokai}
\definecolor{bg}{HTML}{282828} % from https://github.com/kevinsawicki/monokai
%\defaultfontfeatures{}
\newfontfamily\noligsmonofamily[NFSSFamily=noligsmonofamily]{Courier}
\setminted{fontfamily=noligsmonofamily}

\renewcommand{\theFancyVerbLine}{%
    \sffamily \textcolor{Dandelion}{\scriptsize \oldstylenums{\arabic{FancyVerbLine}}}}

\newenvironment{jcode}[3]
{%
    \VerbatimEnvironment
    \begin{listing}[h]%
    \caption{#2}%
    \label{#3}%
    \begin{minted}[xleftmargin=18pt,
                   mathescape,
                   linenos,
                   numbersep=5pt,
                   bgcolor=bg,
                   frame=lines,
                   framesep=2mm,
                   fontsize=\footnotesize]{#1}%
}
{%
    \end{minted}
    \vspace{-25pt}%
    \end{listing}%
}
\renewcommand{\listingscaption}{代码}%from minted
\renewcommand{\listoflistingscaption}{代码列表}% from minted


\newenvironment{myquote}{\begin{quote}\kaishu\zihao{-5}}{\end{quote}}
\newcommand\degree{^\circ}
\newtheorem{thm}{定理}


\begin{document}
\maketitle
\tableofcontents
\listoflistings


\title{Recurrent Neural Network}
\subtitle{Yet Another Grassroot Effort}
\author{Ryan}
\date{\today}
\institute{{解惑者}\\\url{http://www.jiehuozhe.com/}}

\begin{document}

\begin{frame}[plain,t]
\titlepage
\end{frame}


\section{背景介绍}
\begin{frame}
\frametitle{A Frame}
\framesubtitle{Bullet points}
\begin{itemize}
\item First thing
	\begin{itemize}
	\item small point
	\item fine print
	\end{itemize}
\item Second thing
	\begin{enumerate}
	\item point 1
	\end{enumerate}
\item Third thing
	\begin{description}
	\item[Research] the scientific pursuit for knowledge
	\end{description}
\item Third thing
	\begin{description}
	\item[Research] the scientific pursuit for knowledge
	\end{description}

\end{itemize}
\end{frame}

\begin{frame}
\frametitle{A Frame}
\framesubtitle{Bullet points}
\begin{itemize}
\item First thing
	\begin{itemize}
	\item small point
	\item fine print
	\end{itemize}
\item Second thing
	\begin{enumerate}
	\item point 1
	\end{enumerate}
\item Third thing
	\begin{description}
	\item[Research] the scientific pursuit for knowledge
	\end{description}
\end{itemize}
\end{frame}

\subsection{Text}
\begin{frame}
\frametitle{Another Frame}
Lorem ipsum dolor sit amet, consectetur adipisicing elit, sed do eiusmod tempor incididunt ut labore et dolore magna aliqua. Ut enim ad minim veniam, quis nostrud exercitation ullamco laboris nisi ut aliquip ex ea commodo consequat.
\end{frame}

\subsection{Blocks}
\begin{frame}
\frametitle{Blocks}
\begin{definition}[Greetings]
Hello World
\end{definition}

\begin{theorem}[Fermat's Last Theorem]
$a^n + b^n = c^n, n \leq 2$
\end{theorem}

\begin{alertblock}{Uh-oh.}
By the pricking of my thumbs.
\end{alertblock}

\begin{exampleblock}{Uh-oh.}
Something evil this way comes.
\end{exampleblock}

\end{frame}





%%%%%%%%%%%%%%%%%%%%%%%%%%%%%%%%%%%%%%%%%%%%%%%%%%%%%%%%%%%%%%%%%%%%%%%%%%%%%%%%%%%%
% For every picture that defines or uses external nodes, you'll have to
% apply the 'remember picture' style. To avoid some typing, we'll apply
% the style to all pictures.
\tikzstyle{every picture}+=[remember picture]

% By default all math in TikZ nodes are set in inline mode. Change this to
% displaystyle so that we don't get small fractions.
\everymath{\displaystyle}

\begin{frame}
\frametitle{Rigid body dynamics}

\tikzstyle{na} = [baseline=-.5ex]

\begin{itemize}[<+-| alert@+>]
    \item Coriolis acceleration
        \tikz[na] \node[coordinate] (n1) {};
\end{itemize}

% Below we mix an ordinary equation with TikZ nodes. Note that we have to
% adjust the baseline of the nodes to get proper alignment with the rest of
% the equation.
\begin{equation*}
\vec{a}_p = \vec{a}_o+\frac{{}^bd^2}{dt^2}\vec{r} +
        \tikz[baseline]{
            \node[fill=blue!20,anchor=base] (t1)
            {$ 2\vec{\omega}_{ib}\times\frac{{}^bd}{dt}\vec{r}$};
        } +
        \tikz[baseline]{
            \node[fill=red!20, ellipse,anchor=base] (t2)
            {$\vec{\alpha}_{ib}\times\vec{r}$};
        } +
        \tikz[baseline]{
            \node[fill=green!20,anchor=base] (t3)
            {$\vec{\omega}_{ib}\times(\vec{\omega}_{ib}\times\vec{r})$};
        }
\end{equation*}

\begin{itemize}[<+-| alert@+>]
    \item Transversal acceleration
        \tikz[na]\node [coordinate] (n2) {};
    \item Centripetal acceleration
        \tikz[na]\node [coordinate] (n3) {};
\end{itemize}

% Now it's time to draw some edges between the global nodes. Note that we
% have to apply the 'overlay' style.
\begin{tikzpicture}[overlay]
        \path[->]<1-> (n1) edge [bend left] (t1);
        \path[->]<2-> (n2) edge [bend right] (t2);
        \path[->]<3-> (n3) edge [out=0, in=-90] (t3);
\end{tikzpicture}
\end{frame}
%%%%%%%%%%%%%%%%%%%%%%%%%%%%%%%%%%%%%%%%%%%%%%%%%%%%%%%%%%%%%%%%%%%%%%%%%%%%%%%%%%%%

\bibliography{\notesroot/reference/reference.bib}
\end{document}

