\ifx\mlbook\undefined
    \documentclass[10pt,a4paper]{ctexbook}
    \providecommand{\pathroot}{../..}

    \usepackage[CJKbookmarks,colorlinks,linkcolor=red]{hyperref}
    \usepackage{geometry}
    \usepackage{amsmath}

    \geometry{left=3.0cm,right=3.0cm,top=2.5cm,bottom=2.5cm}
    \setmainfont{SimSun}
    \XeTeXlinebreaklocale "zh"
    \XeTeXlinebreakskip = 0pt plus 1pt minus 0.1pt

    \begin{document}
    \setlength{\baselineskip}{20pt}
    \title{模型的优化与评估指标}
    \author{Donald Cheung\\jianzhang9102@gmail.com}
    \date{Sep 21, 2017}
    \maketitle
    \tableofcontents
\fi

\chapter{模型的优化与评估指标}
\section{模型的优化指标}
\subsection{对数损失}

\subsection{均方误差}
对于回归问题来说,最常用的损失函数是均方误差(MSE, mean squared error)\footnote{均方误差也是分类问题中常见的一种损失函数}。它的定义如下:
\[
MSE(y, y')=\frac{\sum\limits_{i=1}^{n}{(y_{i}-y'_{i})^2}}{n}
\]

\subsection{交叉熵}
交叉熵(cross entropy)是常用的评判方法之一。交叉熵刻画了两个概率分布之间的距离,它是分类问题中使用比较广的一种损失函数。

交叉熵是一个信息论中的概念,它原本是用来估算平均编码长度的。

给定两个概率分布$p$和$q$,通过$q$来表示$p$的交叉熵为:
\[
H(p,q)=-\sum\limits_{x}{p(x)\log{q(x)}}
\]
从交叉熵的公式中可以看到交叉熵函数不是对称的($H(p,q) \neq H(q,p)$),它刻画的是通过概率分布$q$来表达概率分布$p$的困难程度。




\section{模型的评估指标}
\subsection{准确率}
\subsection{召回率}
\subsection{F值}
\subsection{AUC}
\subsection{BLEU}
\subsection{NDCG}
\subsection{LogLoss}

\section{统计学相关}
\subsection{皮尔逊相关系数}
\href{https://www.zhihu.com/question/19734616?sort=created}{如何理解皮尔逊相关系数(Pearson Correlation Coefficient)?}

\ifx\mlbook\undefined
    \end{document}
\fi
