\ifx\mlbook\undefined
    \documentclass[10pt,a4paper]{ctexbook}
    \providecommand{\pathroot}{../..}

    \usepackage[CJKbookmarks,colorlinks,linkcolor=red]{hyperref}
    \usepackage{geometry}
    \usepackage{amsmath}
    \usepackage{minted}

    \geometry{left=3.0cm,right=3.0cm,top=2.5cm,bottom=2.5cm}
    \setmainfont{SimSun}
    \XeTeXlinebreaklocale "zh"
    \XeTeXlinebreakskip = 0pt plus 1pt minus 0.1pt

    \begin{document}
    \setlength{\baselineskip}{20pt}
    \title{结巴分词}
    \author{Donald Cheung\\jianzhang9102@gmail.com}
    \date{Oct 26, 2017}
    %\maketitle
    \tableofcontents
\fi

\chapter{结巴分词}

项目地址:\url{https://github.com/fxsjy/jieba/}

\begin{minted}[mathescape,
               numbersep=5pt,
               frame=lines,
               framesep=2mm]{python}
#coding: utf-8
import jieba

words = jieba.cut('我来到北京清华大学', cut_all=True)
print("/".join(words)) #全模式
# 我/来到/北京/清华/清华大学/华大/大学

words = jieba.cut('我来到北京清华大学', cut_all=False)
print('/'.join(words)) #精确模式,默认为该模式
# 我/来到/北京/清华大学

#支持自定义词典
file_name='your/path/to/dictionary'
jieba.load_userdict(file_name) # file_name为自定义词典的路径

# 支持并行分词
jieba.enable_paralle(4)
\end{minted}

\begin{minted}[mathescape,
               numbersep=5pt,
               frame=lines,
               framesep=2mm]{python}
#coding: utf-8
import jieba.posseg as pseg
words = pseg.cut('我来到北京清华大学')
for word, flag in words:
    print('%s %s' % (word, flag))

# output:
# 我 r
# 来到 v
# 北京 ns
# 清华大学 nt


#中文编码
print(len("中文")) # 输出:6 (使用utf8编码)
print(len("中文".decode('utf8'))) # 输出:2(使用utf8编码)


\end{minted}



\ifx\mlbook\undefined
    \end{document}
\fi
