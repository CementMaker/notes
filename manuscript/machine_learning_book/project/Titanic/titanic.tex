\ifx\mlbook\undefined
    \documentclass[10pt,a4paper]{ctexbook}
    \providecommand{\pathroot}{../..}

    \usepackage[CJKbookmarks,colorlinks,linkcolor=red]{hyperref}
    \usepackage{geometry}
    \usepackage{amsmath}
    \usepackage{minted}

    \geometry{left=3.0cm,right=3.0cm,top=2.5cm,bottom=2.5cm}
    \setmainfont{SimSun}
    \XeTeXlinebreaklocale "zh"
    \XeTeXlinebreakskip = 0pt plus 1pt minus 0.1pt

    \begin{document}
    \setlength{\baselineskip}{20pt}
    \title{Titanic}
    \author{Donald Cheung\\jianzhang9102@gmail.com}
    \date{Nov 9, 2017}
    %\maketitle
    \tableofcontents
\fi

\chapter{Titanic}

\section{问题描述}
比赛链接:\url{https://www.kaggle.com/c/titanic}

\subsection{Description}
The sinking of the RMS Titanic is one of the most infamous shipwrecks in history. On April 15, 1912, during her maiden voyage, the Titanic sank after colliding with an iceberg, killing 1502 out of 2224 passengers and crew. This sensational tragedy shocked the international community and led to better safety regulations for ships.

One of the reasons that the shipwreck led to such loss of life was that there were not enough lifeboats for the passengers and crew. Although there was some element of luck involved in surviving the sinking, some groups of people were more likely to survive than others, such as women, children, and the upper-class.

In this challenge, we ask you to complete the analysis of what sorts of people were likely to survive. In particular, we ask you to apply the tools of machine learning to predict which passengers survived the tragedy.

\subsection{Evalution}
It is your job to predict if a passenger survived the sinking of the Titanic or not. 

For each PassengerId in the test set, you must predict a 0 or 1 value for the Survived variable.

Your score is the percentage of passengers you correctly predict. This is known simply as "accuracy”.

The file should have exactly 2 columns:\\
PassengerId (sorted in any order)\\
Survived (contains your binary predictions: 1 for survived, 0 for deceased)\\
PassengerId,Survived\\
892,0\\
893,1\\
894,0\\
Etc.\\

\subsection{数据集}
The data has been split into two groups:

training set (train.csv)

test set (test.csv)

The training set should be used to build your machine learning models. For the training set, we provide the outcome (also known as the “ground truth”) for each passenger. Your model will be based on “features” like passengers’ gender and class. You can also use \href{https://triangleinequality.wordpress.com/2013/09/08/basic-feature-engineering-with-the-titanic-data/}{feature engineering} to create new features.

The test set should be used to see how well your model performs on unseen data. For the test set, we do not provide the ground truth for each passenger. It is your job to predict these outcomes. For each passenger in the test set, use the model you trained to predict whether or not they survived the sinking of the Titanic.

\subsubsection{Data Dictionary}
\begin{table}[H]
\centering
\begin{tabular}{lll}
Variable & Definition & Key \\
survival & Survival & 0 = No, 1 = Yes \\
pclass   & Ticket class & 1 = 1st, 2 = 2nd, 3 = 3rd \\
sex      & Sex \\
Age      & Age in years \\
sibsp    & \# of siblings / spouses aboard the Titanic \\
parch    & \# of parents / children aboard the Titanic \\
ticket   & Ticket number \\
fare     & Passenger fare \\
cabin    & Cabin number \\
embarked & Port of Embarkation & C = Cherbourg, Q = Queenstown, S = Southampton \\
\end{tabular}%
\caption{数据格式}
\label{tab:complex-entropy}
\end{table}

\subsubsection{Variable Notes}
\textbf{pclass}: A proxy for socio-economic status (SES)
1st = Upper
2nd = Middle
3rd = Lower

\textbf{age}: Age is fractional if less than 1. If the age is estimated, is it in the form of xx.5

\textbf{sibsp}: The dataset defines family relations in this way...

Sibling = brother, sister, stepbrother, stepsister

Spouse = husband, wife (mistresses and fiancés were ignored)

\textbf{parch}: The dataset defines family relations in this way...

Parent = mother, father

Child = daughter, son, stepdaughter, stepson

Some children travelled only with a nanny, therefore parch=0 for them.


\section{解决方案}


\ifx\mlbook\undefined
    \end{document}
\fi
