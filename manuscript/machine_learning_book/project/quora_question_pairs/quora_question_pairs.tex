\ifx\mlbook\undefined
    \documentclass[10pt,a4paper]{ctexbook}
    \providecommand{\pathroot}{../..}

    \usepackage[CJKbookmarks,colorlinks,linkcolor=red]{hyperref}
    \usepackage{geometry}
    \usepackage{amsmath}
    \usepackage{float}

    \geometry{left=3.0cm,right=3.0cm,top=2.5cm,bottom=2.5cm}
    \setmainfont{SimSun}
    \XeTeXlinebreaklocale "zh"
    \XeTeXlinebreakskip = 0pt plus 1pt minus 0.1pt

    \usepackage[table, x11names]{xcolor}
    \usepackage{array, booktabs, boldline}
    \usepackage{cellspace}
    \setlength\cellspacetoplimit{4pt}
    \setlength\cellspacebottomlimit{4pt}

    \begin{document}
    \setlength{\baselineskip}{20pt}
    \title{Quora Question Pairs}
    \author{Donald Cheung\\jianzhang9102@gmail.com}
    \date{Sep 28, 2017}
    %\maketitle
    \tableofcontents
\fi

\chapter{Quora Question Pairs}
官方比赛链接:\href{https://www.kaggle.com/c/quora-question-pairs}{Quora Question Pairs}

数据集链接:\url{https://pan.baidu.com/s/1mij3Nza}

\section{问题详情}

\subsection{背景}
Quora是美国版的知乎,月活用户超过1亿,很多人会问很类似的问题。重复问题的出现,会使得信息需求者花费更多的时间去寻找答案,而答主需要对一样的问题解答很多遍。

此数据需要解决的问题就是,判断一对问题是否是重复性问题,也就是说判断两个问题是否是同一个意思。

需要注意的是,数据集中的lable是由人为标注的,也就是说并不是100\%正确的。

\subsection{评估}
用log loss作为评估标准。

\section{实验}

\begin{enumerate}
\item 基于词频的cosine相似度计算:[2017-03-21 21:20]
成绩:0.75402

做法:直接使用sklearn的CountVectorizer,然后获取每个问题的切词列表,统计出频次,计算两个向量的相似度。

TODO: 
\begin{itemize}
\item 基于tf-idf特征的方法
\item 两个问题的BOW向量,直接通过LR训练
\item 获取word的Embedding向量,然后加和取平均,获得问题的向量,然后(1)计算两个向量的相似度;(2)直接拼接两个向量,接入LR训练模型
\item 使用论文`Distributed Representations of Sentences and Documents`中的想法
\item 使用fastText工具
\item 使用LSTM
\item 方法参考:https://www.kaggle.com/c/quora-question-pairs/discussion/30340
\end{itemize}

\end{enumerate}

\ifx\mlbook\undefined
    \end{document}
\fi
