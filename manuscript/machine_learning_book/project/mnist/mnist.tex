\ifx\mlbook\undefined
    \documentclass[10pt,a4paper]{ctexbook}
    \providecommand{\pathroot}{../..}

    \usepackage[CJKbookmarks,colorlinks,linkcolor=red]{hyperref}
    \usepackage{geometry}
    \usepackage{amsmath}
    \usepackage{float}

    \geometry{left=3.0cm,right=3.0cm,top=2.5cm,bottom=2.5cm}
    \setmainfont{SimSun}
    \XeTeXlinebreaklocale "zh"
    \XeTeXlinebreakskip = 0pt plus 1pt minus 0.1pt

    \usepackage[table, x11names]{xcolor}
    \usepackage{array, booktabs, boldline}
    \usepackage{cellspace}
    \setlength\cellspacetoplimit{4pt}
    \setlength\cellspacebottomlimit{4pt}

    \begin{document}
    \setlength{\baselineskip}{20pt}
    \title{MNIST数据集}
    \author{Donald Cheung\\jianzhang9102@gmail.com}
    \date{Sep 22, 2017}
    %\maketitle
    \tableofcontents
\fi

\chapter{MNIST数据集}
\section{数据集介绍}
MNIST数据集是NIST数据集的一个子集,它包含了6000张图片作为训练数据,10000张图片作为测试数据。在MNIST数据集中的每一张图片都代表了0$\sim$9中的一个数字。
图片的大小都为28x28,且数字都会出现在图片的正中间。图\ref{fig:mnist_matrix}展示了一张数字及其所对应的像素矩阵。
\begin{figure}[ht]
    \centering
    \includegraphics[scale=0.4]{\pathroot/project/mnist/images/MNIST-Matrix.png}
    \caption{数字图片及其像素矩阵}
    \label{fig:mnist_matrix}
\end{figure}

在图\ref{fig:mnist_matrix}的左侧显示了一张数字1的图片,而右侧显示了这个图片所对应的像素矩阵。在Yann LeCun教授的网站中(\url{http://yann.lecun.com/exdb/mnist})对MNIST数据集做出了详细的介绍。
MNIST数据集提供了4个下载文件,表\ref{table:mnist_data_url}归纳了下载文件中提供的内容。
\begin{table}[!htb]
\caption{MNIST数据下载地址和内容}
\label{table:mnist_data_url}
\centering
\begin{tabular}{|Sc|c|c|}
\hlineB{2}
\rowcolor{SeaGreen3!30!} \textbf{网址} & \textbf{内容} \\
\hlineB{1.5}
http://yann.lecun.com/exdb/mnist/train-images-idx3-ubyte.gz & 训练数据图片 \\
\hlineB{1.5}
http://yann.lecun.com/exdb/mnist/train-labels-idx1-ubyte.gz & 训练数据答案 \\
\hlineB{1.5}
http://yann.lecun.com/exdb/mnist/t10k-images-idx3-ubyte.gz & 测试数据图片\\
\hlineB{1.5}
http://yann.lecun.com/exdb/mnist/t10k-labels-idx1-ubyte.gz & 测试数据答案 \\
\hlineB{1.5}
\hlineB{2}
\end{tabular}
\end{table}

虽然这个数据集只提供了训练和测试数据,但是为了验证模型训练的效果,一般会从训练数据中划分出一部分数据作为验证(validation)数据。


\ifx\mlbook\undefined
    \end{document}
\fi
