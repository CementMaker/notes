\ifx\mlbook\undefined
    \documentclass[10pt,a4paper]{ctexbook}
    \providecommand{\pathroot}{../..}

    \usepackage[CJKbookmarks,colorlinks,linkcolor=red]{hyperref}
    \usepackage{geometry}
    \usepackage{amsmath}
    \usepackage{minted}

    \geometry{left=3.0cm,right=3.0cm,top=2.5cm,bottom=2.5cm}
    \setmainfont{SimSun}
    \XeTeXlinebreaklocale "zh"
    \XeTeXlinebreakskip = 0pt plus 1pt minus 0.1pt

    \begin{document}
    \setlength{\baselineskip}{20pt}
    \title{语言识别}
    \author{Donald Cheung\\jianzhang9102@gmail.com}
    \date{Nov 9, 2017}
    %\maketitle
    \tableofcontents
\fi

\chapter{语言识别}
参考资料
\begin{itemize}
\item \href{http://blog.csdn.net/rfc2008/article/details/9151755?utm_source=tuicool&utm_medium=referral}{语音识别技术简介}
\end{itemize}

\section{声学模型}
声学模型(acoustic model)是自动语音识别系统的模型中最底层的部分,同时也是自动语音识别系统中最关键的组成单元,声学模型建模的好坏会直接从根本上影响语音识别系统的识别效果和鲁棒性。声学模型实验概率统计的模型对带有声学信息的语音基本单元建立模型,描述其统计特性。通过对声学模型的建模,可以较有效地衡量语音的特征矢量序列和每一个发音模板之间的相似度,可以有助于判断该段语音的声学信息,即语音的内容。语者的语音内容都是由一些基本的语音单元组成,这些基本的语音单元可以是句子、词组、词、音节(syllable)、子音节(Sub-syllable)或者音素等。可见可选择建模的语音单元有不少,通常应该根据具体的应用场景来选择建模的语音单元。在小词汇量的语音识别系统当中通常选用单词作为一个语音单元来建立声学模型,但是当词汇量增多时,需要训练和存储大量的语音数据,很容易出现训练数据不充分或者某些建模单元数据的缺失,导致过拟合问题,影响模型的准确性,甚至缺失某些单词的训练数据,无法对其建模。所以后来出现了使用音节或者子音节建立声学模型的方法。由于一种语言中音节或者子音节比较有限,一般情况下不会出现训练数据不充分或者缺失问题。在词汇量较大的语音识别系统中,这种建模方法要比对单词建模的识别率高。

\section{发音字典}
发音词典是存放所有单词的发音的词典,它的作用是用来连接声学模型和语言模型的。例如,一个句子可以分成若干个单词相连接,每个单词通过查询发音词典得到该单词发音的音素序列。相邻单词的转移概率可以通过语言模型获得,音素的概率模型可以通过声学模型获得。从而生成了这句话的一个概率模型。

\section{语言模型}
随着统计语言处理方法的发展,统计语言模型成为语音识别中语言处理的主流技术,其中统计语言模型有很多种,如N-Gram语言模型、马尔可夫N元模型(Markov N-gram)、指数模型( Exponential Models)、决策树模型(Decision Tree Models)等。而N元语言模型是最常被使用的统计语言模型,特别是二元语言模型(bigram)、三元语言模型(trigram)。

\section{解码器}
解码器是自动语音识别系统的核心模块,其任务是对输入的语音信号,在由语句或者单词序列构成的空间当中,按照一定的优化准则,并且根据声学、语言模型及词典,生成一个用于搜索的状态空间,在该状态空间中索到最优的状态序列,即寻找能够以最大概率输出该信号的句子或者单词序列。


\section{待整理}
百度AI语音识别
\begin{enumerate}
\item 优势:
    \begin{itemize}
    \item 深度语义解析
      \subitem 支持多达35个垂类领域的语义理解定制
    \item 场景识别定制
        \subitem 开发者可根据使用场景,自定义设置识别垂类模型。有音乐、视频、地图、游戏、电商共17个垂类可供选择
    \item 自定义上传语料、训练模型
        \subitem 开发者可以自行上传词库,训练专属识别模型。提交的语料越多、越全,语音识别的效果提升也会越明显
    \end{itemize}

\item 限制:
    \begin{itemize}
    \item 语音识别仅支持以下格式 :pcm(不压缩)、wav(不压缩,pcm编码)、amr(有损压缩格式);8k/16k 采样率 16bit 位深的单声道
    \item 上传需要完整的录音文件,录音文件时长不超过60s
    \end{itemize}
\end{enumerate}

语音识别程序demo完成,如下

语音输入:格式pcm,16k 采样率 16k.pcm 16kx.pcm(见目录data下的数据)

结果展示:(见images下的图片)


后续工作:
\begin{itemize}
\item 格式及编码限制:增加格式编码转换模块
\item 录音文件时长限制:长文件切分模块(匹配时间戳)
\end{itemize}

示例代码:
\begin{minted}[mathescape,
              linenos,
              numbersep=5pt,
              frame=lines,
              framesep=2mm]{python}
#设置应用信息
baidu_server = "https://openai.baidu.com/oauth/2.0/token?"
grant_type = "cliet_credentials"
client_id = "k3Pn3cH67zIC0yLiBOVr8R1" #填写API Key
client_secret = "b3c5dd695c539bd15bdeb01e16dbc20" #填写Secret Key
#合成请求token的URL
url = baidu_server+"grant_type="+grant_type+"&client_id="+client_id+"&client_secret="+client_secret
#获取token
res = urllib2.urlopen(url).read()
data = json.loads(res)
token = data["access_token"]
print token
#设置音频属性,根据百度的要求,采样率必须为16000,压缩格式支持pcm(不压缩)、wav、opus、speex、amr
VOICE_RATE = 16000
WAVE_FILE = "/home/users/xinlongpeng/speech/Baidu_Voice_RetApi_SampleCode/sample/php/test.pcm" #音频文件的路径
USER_ID = "10097316" #用于标识的ID,可以随意设置
WAVE_TYPE = "pcm"
#打开音频文件,并进行编码
f = open(WAVE_FILE, "r")
speech = base64.b64encode(f.read())
size = os.path.getsize(WAVE_FILE)
update = json.dumps({"format":WAVE_TYPE, "rate":VOICE_RATE, 'channel':1,'cuid':USER_ID,'token':token,'speech':speech,'len':size})
headers = { 'Content-Type' : 'application/json' }
url = "http://vop.baidu.com/server_api"
req = urllib2.Request(url, update, headers)
r = urllib2.urlopen(req)
 
t = r.read()
result = json.loads(t)
print result
print result['err_msg']
print word = result['result'][0].encode('utf-8')
if result['err_msg']=='success.':
    word = result['result'][0].encode('utf-8')
    if word!='':
        if word[len(word)-3:len(word)]==',':
            print word[0:len(word)-3]
        else:
            print word
    else:
        print "音频文件不存在或格式错误"
else:
    print "错误"
\end{minted}

\ifx\mlbook\undefined
    \end{document}
\fi
