\ifx\mlbook\undefined
    \documentclass[10pt,a4paper]{ctexbook}
    \providecommand{\pathroot}{../..}

    \usepackage[CJKbookmarks,colorlinks,linkcolor=red]{hyperref}
    \usepackage{geometry}
    \usepackage{amsmath}

    \geometry{left=3.0cm,right=3.0cm,top=2.5cm,bottom=2.5cm}
    \setmainfont{SimSun}
    \XeTeXlinebreaklocale "zh"
    \XeTeXlinebreakskip = 0pt plus 1pt minus 0.1pt

    \begin{document}
    \setlength{\baselineskip}{20pt}
    \title{百度内部机器学习的应用}
    \author{Donald Cheung\\jianzhang9102@gmail.com}
    \date{Oct 30, 2017}
    \maketitle
    \tableofcontents
\fi

\chapter{百度内部机器学习的应用}

\section{LU ACP模型}
\subsection{一期}
\subsubsection{策略背景}
\begin{enumerate}
\item LU-ACP基线已使用DNN模型进行预估,但为连续值DNN,即所使用特征为连续值(统计离散特征维度的广告点击、ACP以及对应的LR-weight作为特征)。
网络结构如图\ref{fig:lu_acp_v1_model}所示:
\begin{figure}[ht]
    \centering
    \includegraphics[height=7cm]{\pathroot/application/baidu/images/LU_ACP_v1_model.png}
    \caption{LU-ACP一期网络结构}
    \label{fig:lu_acp_v1_model}
\end{figure}

\item 相对于连续值DNN模型,大规模离散DNN模型直接使用离散特征作为网络输入,在减少输入信息损失的同时,能更好的利用神经网络的能力学习到特征更优的表示。
\item 大规模离散DNN的成熟应用:
凤巢团队已将大规模离散DNN模型应用于线上点击率预估。
\end{enumerate}

\subsubsection{策略方案}
\begin{enumerate}
\item 引入大规模离散DNN模型,基于MIO-DNN直接使用离散DNN对ACP进行拟合。在离散DNN中直接使用离散特征作为输入,并使用3维向量进行表示。
相对于基线,使用多维度非线性的表示,大幅提升了模型对ACP的刻画能力。
除此之外,我们加入了按时间decay的连续特征如广告点击、历史ACP,并作为连续DNN的输入。
在4天数据上评估AUC提升千分位1.4;
\item 特征优化:在之前的架构上,需要每天多次训练lr模型,并进行连续值特征的计算。这样需要花费大量的机器资源。通过使用MIO-DNN计算,我们去除了基于owlqn的lr模型训练和独立统计的连续值特征,并在离散DNN模型中实时统计和训练,将分离的多个模块进行了统一和优化。通过online-SGD的DNN训练,将原先batch+inc\_batch的训练模型改成了在线学习,也为将来的时效性优化进行了铺垫;
\item 性能优化:我们通过实验,精简DNN网络结构和特征,节约内存和计算性能开销
\end{enumerate}
 
详细参数如下:
网络共4层:

输入层150节点=30个特征*5 (pv、clk、加三维特征表示向量)

隐藏层2层,分别有63、31节点

输出层1节点

\subsection{二期}
\subsubsection{背景概述}
目前LU ACP模型是经典的双模型迭代结构,2个模型均为DNN模型。网络1是个离散DNN模型,基于输入的离散特征进行离散DNN的学习。在网络2中,我们将第一个网络的第一个隐层作为embedding和连续特征(show,ctr)作为输入,进行一个连续DNN的学习,并将DNN的输出作为q值用于LU线上ACP预估。

\begin{figure}[ht]
    \centering
    \includegraphics[height=7cm]{\pathroot/application/baidu/images/LU_ACP_v2_model.png}
    \caption{LU-ACP二期网络结构}
    \label{fig:lu_acp_v2_model}
\end{figure}

\subsubsection{策略升级点}
本次对pserver训练框架升级,使得框架能够支持slot号大于INT16\_MAX的特征正常训练,适应LU模型训练场景。本次升级使得LU ACP模型MF组件覆盖率有效提高,AUC万分位提升6个点。

\ifx\mlbook\undefined
    \end{document}
\fi
