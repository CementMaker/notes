%\documentclass[11pt,a4paper]{article}
\documentclass[UTF8,11pt,a4paper]{ctexart}
\usepackage{fontspec}
\usepackage{geometry}
\geometry{left=3.0cm,right=3.0cm,top=2.5cm,bottom=2.5cm}
\usepackage{color}
\usepackage{graphicx}
\usepackage{animate}
\usepackage[CJKbookmarks,colorlinks,linkcolor=red]{hyperref}

\usepackage{float}

%\usepackage{xunicode}
%\usepackage{xltxtra}
%\usepackage{indentfirst}

%\setromanfont{LiHei Pro}
%\setromanfont{SongTi}
%\setmonofont{Courier New}
%\setromanfont{SimSun}
\setmainfont{SimSun}
\XeTeXlinebreaklocale "zh"
%\XeTeXlinebreakskip = 0pt plus 1pt
\XeTeXlinebreakskip = 0pt plus 1pt minus 0.1pt

\begin{document}
\setlength{\baselineskip}{20pt}
\title{Neural Networks}
\author{Donald Cheung}
\date{Jun 16, 2017}
%\date{\today}
\maketitle
\tableofcontents

\section{神经网络理论基础}
\subsection{FNN}

\subsection{RNN}
RNN理论基础,包括历史、类别、训练算法等。
\subsubsection{BPTT}
\begin{itemize}
\item Hochreiter S, Schmidhuber J. \href{http://web.eecs.utk.edu/~itamar/courses/ECE-692/Bobby_paper1.pdf}{Long short-term memory[J]}. Neural computation, 1997, 9(8): 1735-1780.
\item Bengio Y, Simard P, Frasconi P. \href{http://www-dsi.ing.unifi.it/~paolo/ps/tnn-94-gradient.pdf}{Learning long-term dependencies with gradient descent is difficult[J]}. IEEE transactions on neural networks, 1994, 5(2): 157-166.
\end{itemize}

\begin{itemize}
\item $x_t$是时刻$t$的输入。例如,$x_1$可以是一个one-hot的稀疏向量,表示一个句子中的第二个单词。
\item $s_t$是时刻$t$的隐层状态,是网络的记忆单元。$s_t$的计算依赖于当前时刻的输入以及之前的隐层状态值:$s_t=f(Ux_t+Ws_{t-1})$。函数$f$通常是非线性函数,例如$tanh$或者是$ReLU$。
\item $o_t$是时刻$t$的输出。例如,当我们需要预测一个句子中的下一个词时,$o_t$为词典中每个词的概率,也即$o_t=softmax(Vs_t)$
\item 需要注意的是,RNN中的$U,V,W$参数是共享的。
\end{itemize}




\subsection{CNN}


\section{自然语言处理}
\subsection{Word Embedding}

\subsection{语言模型}
\begin{itemize}
\item \href{http://www.jmlr.org/papers/volume3/bengio03a/bengio03a.pdf}{A Neural Probabilistic Language Model}
\item \href{https://arxiv.org/pdf/1708.02657.pdf}{Which Encoding is the Best for Text Classification in Chinese, English, Japanese and Korean?}
\end{itemize}



\subsection{文本分类}
\subsection{机器翻译}
\subsection{序列标注}


\section{计算机视觉}




\section{RNN技术综述}

\subsection{A Critical Review of Recurrent Neural Networks for Sequence Learning}

\subsection{Neural Turing Machines}
\href{Neural Turing Machines}{https://arxiv.org/abs/1410.5401}

\subsection{Attention}
\href{Attention and Augmented Recurrent Neural Networks}{https://distill.pub/2016/augmented-rnns/}

\subsubsection{Soft Attention}
The concept of attention is the most interesting recent architectural innovation in neural networks.

\subsubsection{Hard Attention}
\href{Inferring Algorithmic Patterns with Stack-Augmented Recurrent Nets}{https://arxiv.org/abs/1503.01007}
\href{Reinforcement Learning Neural Turing Machines - Revised}{https://arxiv.org/abs/1505.00521}
\href{Show, Attend and Tell: Neural Image Caption Generation with Visual Attention}{https://arxiv.org/abs/1502.03044}

\section{People}
\href{Alex Graves}{http://www.cs.toronto.edu/~graves/}
\href{Ilya Sutskever}{http://www.cs.toronto.edu/~ilya/}
\href{Tomas Mikolov}{http://www.rnnlm.org/}

\subsection{Reinforcement Learning}
\href{David Silver}{http://www0.cs.ucl.ac.uk/staff/d.silver/web/Home.html}
\href{Pieter Abbeel}{https://people.eecs.berkeley.edu/~pabbeel/}


\section{文本分类}
\subsection{相关论文}

\begin{itemize}
\item \href{http://www.aclweb.org/old_anthology/D/D15/D15-1167.pdf}{Document Modeling with Gated Recurrent Neural Network for Sentiment Classification}
\item \href{http://www.aclweb.org/anthology/N16-1174}{Hierarchical Attention Networks for Document Classification}
\item \href{http://www.aclweb.org/anthology/P15-1109}{End-to-end Learning of Semantic Role Labeling Using Recurrent Neural Networks}
\end{itemize}

\subsubsection{\href{https://arxiv.org/abs/1408.5882}{Convolutional Neural Networks for Sentence Classification}}

相关资料:
\begin{itemize}
\item 论文代码:\url{https://github.com/yoonkim/CNN_sentence}
\item 代码: \url{https://github.com/dennybritz/cnn-text-classification-tf}
\item \href{http://www.wildml.com/2015/12/implementing-a-cnn-for-text-classification-in-tensorflow/}{Implementing a CNN for Text Classification in TensorFlow}
\item \href{http://www.wildml.com/2015/11/understanding-convolutional-neural-networks-for-nlp/}{Understanding Convolutional Neural Networks for NLP}
\end{itemize}

\subsubsection{\href{https://arxiv.org/abs/1510.03820}{A Sensitivity Analysis of (and Practitioners’ Guide to) Convolutional Neural Networks for Sentence Classification}}

\subsubsection{Convolutional Neural Networks applied to NLP}

The most natural fit for CNNs seem to be classifications tasks, such as Sentiment Analysis, Spam Detection or Topic Categorization. Convolutions and pooling operations lose information about the local order of words, so that sequence tagging as in PoS Tagging or Entity Extraction is a bit harder to fit into a pure CNN architecture (though not impossible, you can add positional features to the input).

[1] Evaluates a CNN architecture on various classification datasets, mostly comprised of Sentiment Analysis and Topic Categorization tasks. The CNN architecture achieves very good performance across datasets, and new state-of-the-art on a few. Surprisingly, the network used in this paper is quite simple, and that's what makes it powerful. The input layer is a sentence comprised of concatenated word2vec word embeddings. That's followed by a convolutional layer with multiple filters, then a max-pooling layer, and finally a softmax classifier. The paper also experiments with two different channels in the form of static and dynamic word embeddings, where one channel is adjusted during training and the other isn't. A similar, but somewhat more complex, architecture was previously proposed in [2]. [6] Adds an additional layer that performs "semantic clustering" to this network architecture.

[4] Trains a CNN from scratch, without the need for for pre-trained word vectors like word2vec or GloVe. It applies convolutions directly to one-hot vectors. The author also proposes a space-efficient bag-of-words-like representation for the input data, reducing the number of parameters the network needs to learn. In [5] the author extends the model with an additional unsupervised "region embedding" that is learned using a CNN predicting the context of text regions. The approach in these papers seems to work well for long-form texts (like movie reviews), but their performance on short texts (like tweets) isn't clear. Intuitively, it makes sense that using pre-trained word embeddings for short texts would yield larger gains than using them for long texts.

Building a CNN architecture means that there are many hyperparameters to choose from, some of which I presented above: Input represenations (word2vec, GloVe, one-hot), number and sizes of convolution filters, pooling strategies (max, average), and activation functions (ReLU, tanh). [7] performs an empirical evaluation on the effect of varying hyperparameters in CNN architectures, investigating their impact on performance and variance over multiple runs. If you are looking to implement your own CNN for text classification, using the results of this paper as a starting point would be an excellent idea. A few results that stand out are that max-pooling always beat average pooling, that the ideal filter sizes are important but task-dependent, and that regularization doesn't seem to make a big different in the NLP tasks that were considered. A caveat of this research is that all the datasets were quite similar in terms of their document length, so the same guidelines may not apply to data that looks considerably different.

[8] explores CNNs for Relation Extraction and Relation Classification tasks. In addition to the word vectors, the authors use the relative positions of words to the entities of interest as an input to the convolutional layer. This models assumes that the positions of the entities are given, and that each example input contains one relation. [9] and [10] have explored similar models.

Another interesting use case of CNNs in NLP can be found in [11] and [12], coming out of Microsoft Research. These papers describe how to learn semantically meaningful representations of sentences that can be used for Information Retrieval. The example given in the papers includes recommending potentially interesting documents to users based on what they are currently reading. The sentence representations are trained based on search engine log data.

Most CNN architectures learn embeddings (low-dimensional representations) for words and sentences in one way or another as part of their training procedure. Not all papers though focus on this aspect of training or investigate how meaningful the learned embeddings are. [13] presents a CNN architecture to predict hashtags for Facebook posts, while at the same time generating meaningful embeddings for words and sentences. These learned embeddings are then successfully applied to another task – recommending potentially interesting documents to users, trained based on clickstream data.

Character-Level CNNs

So far, all of the models presented were based on words. But there has also been research in applying CNNs directly to characters. [14] learns character-level embeddings, joins them with pre-trained word embeddings, and uses a CNN for Part of Speech tagging. [15][16] explores the use of CNNs to learn directly from characters, without the need for any pre-trained embeddings. Notably, the authors use a relatively deep network with a total of 9 layers, and apply it to Sentiment Analysis and Text Categorization tasks. Results show that learning directly from character-level input works very well on large datasets (millions of examples), but underperforms simpler models on smaller datasets (hundreds of thousands of examples). [17] explores to application of character-level convolutions to Language Modeling, using the output of the character-level CNN as the input to an LSTM at each time step. The same model is applied to various languages.

What's amazing is that essentially all of the papers above were published in the past 1-2 years. Obviously there has been excellent work with CNNs on NLP before, as in Natural Language Processing (almost) from Scratch, but the pace of new results and state of the art systems being published is clearly accelerating.

\begin{itemize}
\item Kim, Y. (2014). \href{http://arxiv.org/pdf/1408.5882}{Convolutional Neural Networks for Sentence Classification}. Proceedings of the 2014 Conference on Empirical Methods in Natural Language Processing (EMNLP 2014), 1746–1751.
\item Kalchbrenner, N., Grefenstette, E., \& Blunsom, P. (2014). \href{http://arxiv.org/pdf/1404.2188.pdf}{A Convolutional Neural Network for Modelling Sentences}. Acl, 655–665.
\item Yann N. Dauphin, et al. \href{https://arxiv.org/pdf/1612.08083v1.pdf}{Language Modeling with Gated Convolutional Networks[J]} arXiv preprint arXiv:1612.08083, 2016.
\end{itemize}

%[3] Santos, C. N. dos, & Gatti, M. (2014). Deep Convolutional Neural Networks for Sentiment Analysis of Short Texts. In COLING-2014 (pp. 69–78).
%[4] Johnson, R., & Zhang, T. (2015). Effective Use of Word Order for Text Categorization with Convolutional Neural Networks. To Appear: NAACL-2015, (2011).
%[5] Johnson, R., & Zhang, T. (2015). Semi-supervised Convolutional Neural Networks for Text Categorization via Region Embedding.
%[6] Wang, P., Xu, J., Xu, B., Liu, C., Zhang, H., Wang, F., & Hao, H. (2015). Semantic Clustering and Convolutional Neural Network for Short Text Categorization. Proceedings ACL 2015, 352–357.
%[7] Zhang, Y., & Wallace, B. (2015). A Sensitivity Analysis of (and Practitioners’ Guide to) Convolutional Neural Networks for Sentence Classification,
%[8] Nguyen, T. H., & Grishman, R. (2015). Relation Extraction: Perspective from Convolutional Neural Networks. Workshop on Vector Modeling for NLP, 39–48.
%[9] Sun, Y., Lin, L., Tang, D., Yang, N., Ji, Z., & Wang, X. (2015). Modeling Mention , Context and Entity with Neural Networks for Entity Disambiguation, (Ijcai), 1333–1339.
%[10] Zeng, D., Liu, K., Lai, S., Zhou, G., & Zhao, J. (2014). Relation Classification via Convolutional Deep Neural Network. Coling, (2011), 2335–2344. 
%[11] Gao, J., Pantel, P., Gamon, M., He, X., & Deng, L. (2014). Modeling Interestingness with Deep Neural Networks.
%[12] Shen, Y., He, X., Gao, J., Deng, L., & Mesnil, G. (2014). A Latent Semantic Model with Convolutional-Pooling Structure for Information Retrieval. Proceedings of the 23rd ACM International Conference on Conference on Information and Knowledge Management – CIKM ’14, 101–110. 
%[13] Weston, J., & Adams, K. (2014). # T AG S PACE : Semantic Embeddings from Hashtags, 1822–1827.
%[14] Santos, C., & Zadrozny, B. (2014). Learning Character-level Representations for Part-of-Speech Tagging. Proceedings of the 31st International Conference on Machine Learning, ICML-14(2011), 1818–1826. 
%[15] Zhang, X., Zhao, J., & LeCun, Y. (2015). Character-level Convolutional Networks for Text Classification, 1–9.
%[16] Zhang, X., & LeCun, Y. (2015). Text Understanding from Scratch. arXiv E-Prints, 3, 011102.
%[17] Kim, Y., Jernite, Y., Sontag, D., & Rush, A. M. (2015). Character-Aware Neural Language Models.


\section{RNN应用}

\subsection{语言模型}
\href{Mikolov et al.}{http://www.rnnlm.org}

\subsection{RNN文本生成}
\begin{itemize}
\item \href{http://karpathy.github.io/2015/05/21/rnn-effectiveness/}{The Unreasonable Effectiveness of Recurrent Neural Networks}
\subitem \url{https://github.com/karpathy/char-rnn}
\subitem \url{https://github.com/karpathy/neuraltalk}
\item \href{https://arxiv.org/abs/1412.7755}{MULTIPLE OBJECT RECOGNITION WITH VISUAL ATTENTION}
\item \href{https://arxiv.org/abs/1502.04623}{DRAW: A Recurrent Neural Network For Image Generation}
\item \href{http://www.cs.utoronto.ca/~ilya/pubs/2011/LANG-RNN.pdf}{Generating Text with Recurrent Neural Networks}
\item \href{https://arxiv.org/abs/1308.0850}{Generating Sequences With Recurrent Neural Networks}
\end{itemize}


\subsubsection{Scheduled Sampling}
Paper: \href{https://arxiv.org/pdf/1506.03099.pdf}{Scheduled Sampling for Sequence Prediction with Recurrent Neural Networks}

Scheduled Sampling是一种训练基于RNN的生成模型的算法。原有的基于RNN的生成模型有一个很大的问题, 在训练的时候,使用的训练数据完全为标注的训练数据,例如,给定前一个词预测下一个词的任务中,所用的前一个词总是使用训练数据中的词。但是,在测试的时候,前一个词使用的是在上一个时刻模型生成的词。一般来说,模型生成的词的分布和训练数据中词的分布存在一定的差别。这种差别会导致生成的性能的下降。
Scheduled Sampling算法是一种解决该问题的算法。算法的结构如下图所示。在训练的时候,Scheduled Sampling算法会有一定的概率p选择使用上一个时刻生成的词作为训练的输入。在开始的时候,选择上一个生成的词的概率p一般比较小,因为此时生成的词很大的可能是错误的。随着训练的进行,生成的质量逐渐提高,概率p也逐渐提高。

百度内部wiki参考:\url{http://wiki.baidu.com/pages/viewpage.action?pageId=142379918}


\subsection{Sequence to Sequence Leanring}

\begin{itemize}
\item \href{https://arxiv.org/pdf/1409.3215.pdf}{Sequence to Sequence Learning with Neural Networks}
\item \href{https://arxiv.org/abs/1409.0473}{Neural Machine Translation by Jointly Learning to Align and Translate}
\item \href{https://arxiv.org/abs/1503.08895}{End-To-End Memory Networks}
\item \href{https://arxiv.org/pdf/1603.01354v5.pdf}{End-to-end Sequence Labeling via Bi-directional LSTM-CNNs-CRF}
\item \href{https://arxiv.org/pdf/1508.01991v1.pdf}{Bidirectional LSTM-CRF Models for Sequence Tagging}
\item Cho K, Van Merriënboer B, Gulcehre C, et al. \href{http://arxiv.org/pdf/1406.1078}{Learning phrase representations using RNN encoder-decoder for statistical machine translation[J]}. arXiv preprint arXiv:1406.1078, 2014.
\end{itemize}


\subsection{transcribe speech to text}
\href{http://proceedings.mlr.press/v32/graves14.pdf}{Towards End-to-End Speech Recognition with Recurrent Neural Networks}

\subsection{RNN手写识别}
\href{http://www6.in.tum.de/Main/Publications/Liwicki2007a.pdf}{《Liwicki M, Graves A, Bunke H, et al. A novel approach to on-line handwriting recognition based on bidirectional long short-term memory》}

点评:RNN是时间上的建模,手写字体识别是随着时间的推移字体状态发生了改变,每一个字都有一类状态转移过程,所以特别适合在线手写字体的识别。当然,离线的字体识别已经失去了时间信息,由于CNN是空间上的建模,实现点、线、区域、整个物体、整幅图像的特征提取,此时使用CNN更合适。CNN和RNN是搞OCR,OCR有是在线教育领域必须的技术点,无论是以题搜题,还是在写手写识别。

\href{https://arxiv.org/abs/1308.0850}{Generating Sequences With Recurrent Neural Networks}
\url{http://www.cs.toronto.edu/~graves/handwriting.cgi?text=My+name+is+Jian+Zhang.\&style=\&bias=0.15\&samples=3}
\url{https://github.com/szcom/rnnlib}

\subsection{computer vision}
\href{https://arxiv.org/abs/1406.6247}{Recurrent Models of Visual Attention}


\subsection{video classification}
\href{https://arxiv.org/abs/1411.4389}{Long-term Recurrent Convolutional Networks for Visual Recognition and Description}

\subsection{image captioning}
\href{https://arxiv.org/pdf/1411.4555.pdf}{Show and Tell: A Neural Image Caption Generator}

\subsection{video captioning}
\href{https://arxiv.org/abs/1505.00487}{Sequence to Sequence -- Video to Text}


\subsection{visual question answering}
\href{https://arxiv.org/abs/1505.02074}{Exploring Models and Data for Image Question Answering}


\subsection{RNN动作识别}
\href{http://t.cn/RU8EKNZ}{《Action Recognition using Visual Attention》Shikhar Sharma, Ryan Kiros, Ruslan Salakhutdinov (2015)}

GitHub: \url{https://github.com/kracwarlock/action-recognition-visual-attention}

点评:RNN是时间上的建模,动作识别依赖于图像序列,所以特别适合做动作识别。 CNN和RNN将更好提高学习效率!

\section{Neural Machine Translation}

对于Encoder-Decoder框架来说,
\begin{itemize}
\item encoder读取一个vector序列$X=(x_1,...,x_{T_x})$,并转换成一个向量$c$。最常用的是使用RNN,使得 $h_t=f(x_t,h_{t-1})$和$c=q({h_1,...,h_{T_x}})$
\subitem 其中$f$可以是$LSTM$,$q({h_1,...,h_T})=h_T$
\item decoder通常用于根据上下文向量$c$和所有之前已经预测的$\{y_1,...,y_{t'-1}\}$来预测下一个词$y_{t'}$。也即decoder为以下条件概率
\begin{equation}
p({\rm y})=\prod_{t=1}^{T}p(y_t|{y_1, \cdots, y_{t-1}},c),
\end{equation}
对于RNN来说,每个条件概率可以为: $p(y_t|{y_1, \cdots, y_{t-1}},c)=g(y_{t-1},s_t,c)$

\item 在基于神经网络的机器翻译模型架构中,条件概率定义为: 
\begin{equation}p(y_i|y_1, \cdots, y_{i-1},x)=g(y_{i-1},s_i,c_i)\end{equation}
其中$s_i$为RNN在时刻$i$时的隐层状态,\begin{equation}s_i=f(s_{i-1},y_{i-1},c_{i})\end{equation}
上下文$c_i$为加权和\begin{equation}c_i=\sum\limits_{j=1}^{T_x}\alpha_{ij}h_{j}\end{equation}
其中,权重系数$\alpha_{ij}$计算方式如下
\begin{equation} \alpha_{ij}=\frac{exp(e_{ij})}{\sum_{k=1}^{T_x}exp(e_{ik})} \end{equation}
\begin{equation} e_{ij}=a(s_{i-1},h_j) \end{equation}


\end{itemize}

@ARTICLE{Britz:2017,
  author          = {{Britz}, Denny and {Goldie}, Anna and {Luong}, Thang and {Le}, Quoc},
  title           = "{Massive Exploration of Neural Machine Translation Architectures}",
  journal         = {ArXiv e-prints},
  archivePrefix   = "arXiv",
  eprinttype      = {arxiv},
  eprint          = {1703.03906},
  primaryClass    = "cs.CL",
  keywords        = {Computer Science - Computation and Language},
  year            = 2017,
  month           = mar,
}




%\\$x=\begin{matrix} 0 & 1 \end{matrix}$
%\includegraphics[height=高度][angle=旋转角度]{图片文件名}
下面是一张图片
%\includegraphics[width=0.8\linewidth]{picture/cnn_for_text_classification.png}

\begin{figure}[ht]
\centering
%\includegraphics[scale=0.6]{picture/cnn_for_text_classification.png}
\caption{宋赵爽在《周髀算经》注中作的弦图(仿制),该图给出了勾股定理的一个极具对称美的证明。} 
\label{fig:cnn}
\end{figure}:

\begin{table}[H]
\begin{tabular}{|rrr|}
\hline
直角边 $a$ & 直角边 $b$ & 斜边 $c$\\
\hline
3 & 4 & 5 \\
5 & 12 & 13 \\
\hline
\end{tabular}%
\qquad
($a^2 + b^2 = c^2$)
\end{table}

上面是一张图片
%\animategraphics[height=2.8in,autoplay,controls]{12}{picture/tmp/Convolution_schematic-}{0}{8}
%\animategraphics[height=2.8in,autoplay,controls]{12}{picture/Convolution_schematic.gif}{0}{39}

\section{训练算法}

\subsection{RMSProp}
\href{https://arxiv.org/abs/1502.04390}{Equilibrated adaptive learning rates for non-convex optimization}

\subsection{Adam}
per-parameter adaptive learning rate methods


\section{强化学习}

\subsection{深度强化学习}
\begin{itemize}
\item \href{https://arxiv.org/pdf/1704.03732.pdf}{Learning from Demonstrations for Real World Reinforcement Learning}
\end{itemize}


\section{模型评估指标}
\subsection{ROC/AUC}
\subsection{KS值}
风控模型

\section{深度学习框架}
\subsection{Paddle}

百度wiki:\href{http://wiki.baidu.com/pages/viewpage.action?pageId=38550738}{Paddle代码帮助}


\section{工具}
\subsection{Xapian}
\subsection{Elasticsearch}

\end{document} 

