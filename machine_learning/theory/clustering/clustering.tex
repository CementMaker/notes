\ifx\mlbook\undefined
    \providecommand{\pathroot}{../..}

    \title{聚类}
    \author{Donald Cheung\\jianzhang9102@gmail.com}
    \date{\today\footnote{文档编写开始于2018年03月02日}}

    \documentclass[a4paper,10pt]{ctexbook}
\usepackage{xeCJK}
\usepackage{fontspec}
\usepackage{minted}
\usepackage[CJKbookmarks,colorlinks,linkcolor=red]{hyperref}
\usepackage{geometry}
\usepackage{amsmath}
\usepackage[format=hang,font=small,textfont=it]{caption}
\usepackage{float}
\usepackage{subfigure}
\usepackage[nottoc]{tocbibind}
\usepackage{bm}

\setmainfont{Times New Roman}
\setsansfont{Helvetica}
\setmonofont{Courier New}
\setCJKmainfont[BoldFont={SimHei},ItalicFont={SimHei}]{SimSun}
\setCJKsansfont{SimSun}
\setCJKmonofont{SimSun}

\geometry{left=3.0cm,right=3.0cm,top=2.5cm,bottom=2.5cm}

\newenvironment{myquote}{\begin{quote}\kaishu\zihao{-5}}{\end{quote}}
\newcommand\degree{^\circ}
\newtheorem{thm}{定理}

\bibliographystyle{plain}

\begin{document}
\maketitle
\tableofcontents

\fi

\chapter{聚类}
In the clustering problem, we are given a training set $\{x^{(1)}, \cdots, x^{(m)}\}$,
and want to group the data into a few cohesive ``clusters''.
Here, $x^{(i)} \in R_n$ as usual; but no labels $y^{(i)}$ are given.
So, this is an unsupervised learning problem.

\section{k-means}
The k-means clustering algorithm is as follows:
\begin{enumerate}
    \item Initialize \textbf{cluster centroids} $\mu_1, \mu_2, \cdots, \mu_k \in R_n$ randomly.
    \item Repeat until convergence: \{

            \qquad For every $i$, set
            \[
                c^{(i)} := \arg\min\limits_{j}{||x^{(i)} - \mu_j||}^2.
            \]

            \qquad For each $j$, set
            \[
                \mu_j := \frac{\sum_{i=1}^{m}{ 1\{c^{(i)}=j\}x^{(i)} }} {\sum_{i=1}^{m}{ 1\{c^{(i)}=j\} }}
            \]
        \}
\end{enumerate}

Is the k-means algorithm guaranteed to converge? Yes it is, in a certain sense.

In particular, let us define the \textbf{distortion function} to be:
\[
    J(c, \mu) = \sum_{i=1}^{m}{||x^{(i)} - \mu_{c^{(i)}}||}
\]

It can be shown that k-means is exactly coordinate descent on $J$.


\ifx\mlbook\undefined
    \bibliography{\pathroot/reference/reference.bib}
\end{document}

\fi
