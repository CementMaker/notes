\ifx\mlbook\undefined
    \providecommand{\pathroot}{../..}

    \title{EM算法}
    \author{Donald Cheung\\jianzhang9102@gmail.com}
    \date{\today\footnote{文档编写开始于2018年03月02日}}

    \documentclass[a4paper,10pt]{ctexbook}
\usepackage{xeCJK}
\usepackage{fontspec}
\usepackage{minted}
\usepackage[CJKbookmarks,colorlinks,linkcolor=red]{hyperref}
\usepackage{geometry}
\usepackage{amsmath}
\usepackage[format=hang,font=small,textfont=it]{caption}
\usepackage{float}
\usepackage{subfigure}
\usepackage[nottoc]{tocbibind}
\usepackage{bm}

\setmainfont{Times New Roman}
\setsansfont{Helvetica}
\setmonofont{Courier New}
\setCJKmainfont[BoldFont={SimHei},ItalicFont={SimHei}]{SimSun}
\setCJKsansfont{SimSun}
\setCJKmonofont{SimSun}

\geometry{left=3.0cm,right=3.0cm,top=2.5cm,bottom=2.5cm}

\newenvironment{myquote}{\begin{quote}\kaishu\zihao{-5}}{\end{quote}}
\newcommand\degree{^\circ}
\newtheorem{thm}{定理}

\bibliographystyle{plain}

\begin{document}
\maketitle
\tableofcontents

\fi

\chapter{EM算法}

\section{Jensen不等式}

%\begin{thm}{Jensen不等式}
\begin{thm}
设$f$是一个下凸函数,$X$是一个随机变量,于是有:
\[
    E[f(X)] \ge f(EX)
\]
此外,如果$f$是一个严格下凸函数,则有 $E[f(X)] = f(EX)$ 成立当且仅当 $X=E[X]$ 的概率为1(也就是说,当$X$为一个常量时)。
对于上凸函数,Jensen不等式同样成立,只是大小的方向改变了。
\end{thm}


\section{EM算法}
    Suppose we have an estimation problem in which we have a training set $\{x^{(1)}, \cdots, x^{(m)}\}$ consisting of $m$ independent examples.
We wish to fit the parameters of a model $p(x, z)$ to the data, where the likelihood is given by
\begin{align*}
    \ell (\theta) &= \sum\limits_{i=1}^{m}{\log p(x;\theta)} \\
                  &= \sum\limits_{i=1}^{m}{\log \sum\limits_{z}{p(x,z; \theta)}}.
\end{align*}

    But, explicitly finding the maximum likelihood estimates of the parameters $\theta$ may be hard.
Here, the $z^{(i)}$'s are the latent random variables;
and it is often the case that if the $z^{(i)}$'s were observed, then maximum likelihood estimation would be easy.

    In such a setting, the EM algorithm gives an efficient method for maximum likelihood estimation.
Maximizing $\ell (\theta)$ explicitly might be difficult, and our strategy will be to instead
repeatedly construct a lower-bound on $\ell$ (E-step), and then optimize that lower-bound (M-step).

    For each $i$, let $Q_i$ be some distribution over the $z$'s($\sum_{z}{Q_{i}{(z)} = 1}, Q_i(z) \ge 0)$.
Consider the following:
\begin{align}
    \sum\limits_{i}{\log p(x^{(i)}; \theta)}
        &= \sum\limits_{i}{\log \sum\limits_{z^{(i)}}{p(x^{(i)}, z^{(i)}; \theta)} } \\
        &= \sum\limits_{i}{\log \sum\limits_{z^{(i)}}{Q_i(z^{(i)}) \frac{p(x^{(i)}, z^{(i)}; \theta)}{Q_i(z^{(i)})}}} \\
        &\ge \sum\limits_{i}{\sum\limits_{z^{(i)}}{Q_i(z^{(i)}) \log\frac{p(x^{(i)}, z^{(i)}; \theta)}{Q_i(z^{(i)})}}} \label{eq:em_lower}
\end{align}

    Now, for any set of distributions $Q_i$, the formula \ref{eq:em_lower} gives a lower-bound on $\ell(\theta)$.
There're many possible choices for the $Q_i$'s. Which should we choose?  Well, if we have some current guess
$\theta$ of the parameters, it seems natural to try to make the lower-bound tight at that value of $\theta$.
I.e., we'll make the inequality above hold with equality at our particular value of $\theta$.
(We'll see later how this enables us to prove that $\ell(\theta)$ increases monotonically with
successsive iterations of EM.)

    To make the bound tight for a particular value of $\theta$, we need for the step involving Jensen's
inequality in our derivation above to hold with equality. For this to be true, we know it is sufficient
that the expectation be taken over a ``constant"-valued random variable. I.e., we require that
\[
    \frac{p(x^{(i)},z^{(i)};\theta)}{Q_i(z^{(i)})}=c
\]
for some constant $c$ that does not depend on $z^{(i)}$. This is easily accomplished by choosing
\[
    Q_i(z^{(i)}) \propto p(x^{(i)}, z^{(i)}; \theta).
\]
Actually, since we know $\sum_{z}{Q_i(z^{(i)})} = 1$ (because it is a distribution), this further tells us that
\begin{align*}
    Q_i(z^{(i)}) &= \frac{p(x^{(i)},z^{(i)}; \theta)}{\sum_{z}{p(x^{(i)},z^{(i)};\theta)}} \\
                 &= \frac{p(x^{(i)},z^{(i)}; \theta)}{p(x^{(i)};\theta)} \\
                 &= p(z^{(i)} | x^{(i)}; \theta)
\end{align*}
Thus, we simply set the $Q_i$'s to be the posterior distribution of the $z^{(i)}$'s given $x^{(i)}$ and the setting of the parameters $\theta$.

    Now, for this choice of the $Q_i$'s, equation \ref{eq:em_lower} gives a lower-bound on the
$\log$-likelihood $\ell$ that we're trying to maximize. This is the E-step.
In the M-step of the algorithm, we then maximize our formula
in equation \ref{eq:em_lower} with respect to the parameters to obtain a new setting of the $\theta$'s.
Repeatedly carrying out these two steps gives us the EM algorithm, which is as follows:

    \qquad Repeat until convergence \{

        \qquad\qquad (E-step) For each $i$, set
                \[
                    Q_i(z^{(i)}) := p(z^{(i)}|x^{(i)}; \theta).
                \]

        \qquad\qquad (M-step) Set
                \[
                    \theta := \arg\max \limits_{\theta} \sum\limits_{i} \sum\limits_{z^{(i)}}
                                Q_i(z^{(i)})\log\frac{p(x^{(i)}, z^{(i)}; \theta)}{Q_i(z^{(i)})}
                \]

    \qquad \}

    How we know if this algorithm will converge? Well, suppose $\theta^{(t)}$ and $\theta^{(t+1)}$ are the
parameters from two successive iterations of EM. We will now prove that
$\ell(\theta^{(t)}) \le \ell(\theta^{(t+1)})$, which shows EM always monotonically improves the $\log$-likelihood.
The key to showing this result lies in our choice of the $Q_i$'s.
Specifically, on the iteration of EM in which the parameters had started out as $\theta(t)$,
we would have chosen $Q_i^{(t)}(z^{(i)}) := p(z^{(i)}|x^{(i)};\theta^{(t)})$. We saw earlier that this choice
ensures that Jensen's inequality, as applied to get Equation \ref{eq:em_lower}, holds with equality, and hence 
\[
    \ell(\theta^{(t)}) = \sum\limits_{i} \sum\limits_{z^{(i)}}
            Q_i^{(t)}(z^{(i)}) \log\frac{p(x^{(i)},z^{(i)};\theta^{(t)})}{Q_i^{(t)}(z^{(i)})}
\]
The parameters $\theta^{(t+1)}$ are then obtained by maximizing the right hand side of the equation above.
Thus,
\begin{align*}
    \ell(\theta^{(t+1)}) & \ge \sum\limits_{i} \sum\limits_{z^{(i)}}
                            Q_i^{(t)}(z^{(i)}) \log\frac{p(x^{(i)},z^{(i)};\theta^{(t+1)})}{Q_i^{(t)}(z^{(i)})} \\
                         & \ge \sum\limits_{i} \sum\limits_{z^{(i)}}
                            Q_i^{(t)}(z^{(i)}) \log\frac{p(x^{(i)},z^{(i)};\theta^{(t)})}{Q_i^{(t)}(z^{(i)})} \\
                         &= \ell(\theta^{(t)})
\end{align*}

    Hence, EM causes the likelihood to converge monotonically. In our description of the EM algorithm,
we said we'd run it until convergence. Given the result that we just showed, one reasonable convergence
test would be to check if the increase in $\ell(\theta)$ between successive iterations is smaller than
some tolerance parameter, and to declare convergence if EM is improving $\ell(\theta)$ too slowly.

\textbf{Remark.} If we define
\[
    J(Q, \theta) = \sum\limits_{i} \sum\limits_{z^{(i)}}
                            Q_i(z^{(i)}) \log\frac{p(x^{(i)}, z^{(i)}; \theta)}{Q_i(z^{(i)})}
\]
the we know $\ell(\theta) \ge J(Q,\theta)$ from our previous derivation.
The EM can also be viewed a coordinate ascent on $J$, in which the E-step maximizes it with respect to $Q$,
and the M-step maximizes it with respect to $\theta$.

\ifx\mlbook\undefined
    \bibliography{\pathroot/reference/reference.bib}
\end{document}

\fi
