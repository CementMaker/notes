\ifx\mlbook\undefined
    \documentclass[a4paper,10pt]{ctexbook}
    \providecommand{\pathroot}{../..}

    \usepackage{xeCJK}
    \usepackage{fontspec}
    \usepackage{minted}
    \usepackage[CJKbookmarks,colorlinks,linkcolor=red]{hyperref}
    \usepackage{geometry}
    \usepackage{amsmath}
    \usepackage[format=hang,font=small,textfont=it]{caption}
    \usepackage{float}

    \setmainfont{Times New Roman}
    \setsansfont{Helvetica}
    \setmonofont{Courier New}
    \setCJKmainfont[BoldFont={SimHei},ItalicFont={SimHei}]{SimSun}
    \setCJKsansfont{SimSun}
    \setCJKmonofont{SimSun}

    \geometry{left=3.0cm,right=3.0cm,top=2.5cm,bottom=2.5cm}

    \newenvironment{myquote}{\begin{quote}\kaishu\zihao{-5}}{\end{quote}}
    \newcommand\degree{^\circ}

    %\XeTeXlinebreaklocale "zh"
    %\XeTeXlinebreakskip = 0pt plus 1pt minus 0.1pt
    %\setlength{\baselineskip}{20pt}

    \title{微积分}
    \author{Donald Cheung\\jianzhang9102@gmail.com}
    \date{\today \footnote{文档编写开始于2017年11月10日}}


    \begin{document}
    \maketitle
    \tableofcontents
\fi

\chapter{微积分}
\href{http://dataunion.org/20714.html}{寻找最优参数解:最速下降法,牛顿下降法,阻尼牛顿法,拟牛顿法DFP/BFGS}

\section{梯度下降法}
这一节用来介绍常用的数学优化算法--梯度下降法,以及相关变形。

\href{https://arxiv.org/abs/1609.04747}{An overview of gradient descent optimization algorithms}

\href{http://ruder.io/optimizing-gradient-descent}{An overview of gradient descent optimization algorithms}

这篇文章简单介绍了一下梯度下降法,里面有一些素材可以使用:\href{https://www.analyticsvidhya.com/blog/2017/03/introduction-to-gradient-descent-algorithm-along-its-variants}{Introduction to Gradient Descent Algorithm (along with variants) in Machine Learning}

\begin{minted}{python}
w = 0
for i in range(0, n):
    w = w - 0.1 * f'(w)
\end{minted}

\begin{table}[H]
    \begin{tabular}{|l|l|l|l|l|}
        \hline
                                & \textbf{batch} & \textbf{mini-batch} & \textbf{Stochastic} & \textbf{Online} \\
        \hline
        \textbf{训练集}         & 固定       & 固定               & 固定               & 实时更新 \\
        \hline
        \textbf{单次迭代样本数} & 整个训练集 & 训练集的子集       & 单个样本           & 根据具体算法定 \\
        \hline
        \textbf{算法复杂度}     & 高         & 一般               & 低                 & 低 \\
        \hline
        \textbf{时效性}         & 低         & 一般(delta 模型) & 一般(delta 模型) & 高 \\
        \hline
        \textbf{收敛性}         & 稳定       & 较稳定             & 不稳定             & 不稳定 \\
        \hline
    \end{tabular}
\end{table}
        

\subsection{基本原理}

\subsection{相关变形}

\subsubsection{Batch gradient descent}
每次迭代的梯度方向计算由所有训练样本共同投票决定,batch GD的损失函数是:

\[
    J(\theta)=\frac{1}{2m}{\sum_{i=1}^{m}{(h_{\theta}(x^{(i)}) - y^{(i)})^2}}
\]

训练算法为:
\begin{align*}
    & repeate \{ \\
    &    \theta = \theta - \alpha \frac{1}{m}{\sum_{i=1}^{m}{(h_{\theta}(x^{(i)}) - y^{(i)})x^{(i)}}} \\
    & \}
\end{align*}
什么意思呢,batch GD算法是计算损失函数在整个训练集上的梯度方向,沿着该方向搜寻下一个迭代点。``batch''的含义是训练集中所有样本参与每一轮迭代。


\subsubsection{Mini-batch gradient descent}
batch GD每一轮迭代需要所有样本参与,对于大规模的机器学习应用,经常有billion级别的训练集,计算复杂度非常高。因此,有学者就提出,反正训练集只是数据分布的一个采样集合,我们能不能在每次迭代只利用部分训练集样本呢?这就是mini-batch算法。

假设训练集有$m$个样本,每个mini-batch(训练集的一个子集)有$b$个样本,那么,整个训练集可以分成$\frac{m}{b}$个mini-batch。


\subsubsection{Stochastic gradient descent}
随机梯度下降算法(SGD)是mini-batch GD的一个特殊应用。SGD等价于$b=1$的mini-batch GD。即,每个mini-batch中只有一个训练样本。

\subsubsection{Online}
随着互联网行业的蓬勃发展,数据变得越来越``廉价''。很多应用有实时的,不间断的训练数据产生。在线学习(Online Learning)算法就是充分利用实时数据的一个训练算法。

Online GD与mini-batch GD/SGD的区别在于,所有训练数据只用一次,然后丢弃。这样做的好处是可以获得模型的变化趋势。比如搜索广告的点击率(CTR)预估模型,网民的点击行为会随着时间改变。用batch算法(每天更新一次)一方面耗时较长(需要对所有历史数据重新训练);另一方面,无法及时反馈用户的点击行为迁移。而Online Leaning的算法可以实时的最终网民的点击行为迁移。


\subsection{Gradient descent optimization algorithms}
\subsubsection{Momentum}
\subsubsection{Nesterov accelerated gradient}
\subsubsection{Adagrad}




\subsubsection{Adadelta}
原始论文:\href{https://arxiv.org/abs/1212.5701}{ADADELTA: An Adaptive Learning Rate Method}






\subsubsection{RMSprop}
\subsubsection{Adam}
\subsubsection{AdaMax}
\subsubsection{Nadam}
\subsubsection{Visualization of algorithms}
\subsubsection{Which optimizer to choose?}

\subsection{Parallelizing and distributing SGD}
\subsubsection{Hogwild!}
\subsubsection{Downpour SGD}
\subsubsection{Delay-tolerant Algorithms for SGD}
\subsubsection{TensorFlow}

\subsubsection{Elastic Averaging SGD}
\subsection{Additional strategies for optimizing SGD}
\subsubsection{Shuffling and Curriculum Learning}
\subsubsection{Batch normalization}

\subsubsection{Early Stopping}
\subsubsection{Gradient noise}


\section{牛顿迭代法}
\subsection{基本原理}
假设 $f(x) = 0$ 为待求解方程,利用传统方法求解,牛顿法求解方程的公式:
\[
f(x_0 + \Delta x) = f(x_0) + f'(x_0)\Delta x
\]
即
\[
    f(x) = f(x_0) + f'(x_0)(x - x_0)
\]

令 $f(x) = 0$,我们能够得到迭代公式:
\[
    x = x_0 - \frac{f(x_0)}{f'(x_0)} \Rightarrow x_{n+1} = x_n - \frac{f(x_n)}{f'(x_n)}
\]
通过逐次迭代,牛顿法将逐步逼近最优值,也就是方程的解。

解决 $f(x) = 0$ 的问题,我们用了一阶泰勒展开:
\[
    f(x) = f(x_0) + f'(x_0)*(x-x_0) + o((x-x_0))
\]
去掉末位高阶展开项,代入 $x = x_0 + \Delta x$,得到:
\[
    f(x) = f(x_0 + \Delta x) = f(x_0) + f'(x_0)\Delta x
\]


那么,要解决 $f'(x) = 0$ 的问题,我们就需要二阶泰勒展开:
\[
    f(x) = f(x_0) + f'(x_0)*(x-x_0) + \frac{f''(x_0)}{2}*(x-x_0)^2 + o((x-x_0)^2)
\]
去掉末位高阶展开项,代入 $x = x_0+\Delta x$,得到:
\[
    f(x) = f(x_0+\Delta x) = f(x_0) + f'(x_0) \Delta x + \frac{f''(x_0) (\Delta x)^2}{2}
\]
求导计算: $f'(x) = f'(x_0+\Delta x) = 0$,得到:
\[
    [f(x_0) + f'(x_0)(x−x_0) + \frac{f''(x_0)(x−x_0)^2}{2}]′ = 0
\]
整理:
\begin{align*}
    & f'(x_0) + f''(x_0)(x−x_0) = 0 \\
    & x = x_0 − \frac{f'(x_0)}{f''(x_0)} \Rightarrow  x_{n+1} = x_n - \frac{f'(x_n)}{f''(x_n)}
\end{align*}

\subsection{拟牛顿条件}

\section{拟牛顿法(Quasi-Newton)}
\subsection{DFP 算法}
\subsection{BFGS算法}
\subsection{L-BFGS 算法}


\section{共轭梯度法(Conjugate Gradient)}
学习资料
\begin{enumerate}
\item《最优化理论与方法》 袁亚湘 孙文瑜
\item http://blog.csdn.net/nocml/article/details/8287466
\item Updating Quasi-Newton Matrices with Limited Storage , Jorge Nocedal
\item Nonlinear Programming, second edition, Dimitri P. Bertsekas
\item《广告数据上的大规模机器学习》  夏粉
\end{enumerate}

\ifx\mlbook\undefined
    \end{document}
\fi
