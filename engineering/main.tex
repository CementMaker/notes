\documentclass[utf8,a4paper,10pt]{ctexbook}

\usepackage{xeCJK}
\usepackage{fontspec}
\usepackage[CJKbookmarks,colorlinks,linkcolor=red]{hyperref}
\usepackage{geometry}

\setmainfont{Times New Roman}
\setsansfont{Helvetica}
\setmonofont{Courier New}
\setCJKmainfont[BoldFont={SimHei},ItalicFont={SimHei}]{SimSun}
\setCJKsansfont{SimSun}
\setCJKmonofont{SimSun}

\geometry{left=3.0cm,right=3.0cm,top=2.5cm,bottom=2.5cm}

\title{工程}
\author{Donald Cheung\thanks{corresponding author}\\jianzhang9102@gmail.com}
\date{\today\footnote{文档编写开始于2017年12月28日}}

\begin{document}
\maketitle
\tableofcontents

\chapter{分布式}

\section{工具}

\subsection{ZooKeeper}
为了防止分布式系统中的多个进程之间相互干扰,我们需要一种分布式协调技术来对这些进程进行调度。
而这个分布式协调技术的核心就是来实现这个分布式锁。ZooKeeper是分布式锁的一种实现。

ZooKeeper是一种为分布式应用所设计的高可用、高性能且一致的开源协调服务,
它提供了一项基本服务:分布式锁服务。
由于ZooKeeper的开源特性,后来我们的开发者在分布式锁的基础上,摸索了出了其他的使用方法:
配置维护、组服务、分布式消息队列、分布式通知/协调等。

注意:ZooKeeper性能上的特点决定了它能够用在大型的、分布式的系统当中。
从可靠性方面来说,它并不会因为一个节点的错误而崩溃。
除此之外,它严格的序列访问控制意味着复杂的控制原语可以应用在客户端上。
ZooKeeper在一致性、可用性、容错性的保证,也是ZooKeeper的成功之处,
它获得的一切成功都与它采用的协议 --- Zab协议是密不可分的。

前面提到了那么多的服务,比如分布式锁、配置维护、组服务等,那它们是如何实现的呢?
ZooKeeper在实现这些服务时,首先它设计一种新的数据结构 --- Znode,
然后在该数据结构的基础上定义了一些原语,也就是一些关于该数据结构的一些操作。
有了这些数据结构和原语还不够,因为ZooKeeper是工作在一个分布式的环境下,
服务是通过消息以网络的形式发送给我们的分布式应用程序,所以还需要一个通知机制 --- Watcher机制。
那么总结一下,ZooKeeper所提供的服务主要是通过:数据结构 + 原语 + watcher机制,三个部分来实现的。

\subsubsection{相关学习材料}
\begin{enumerate}
    \item \href{http://www.cnblogs.com/sunddenly/p/4033574.html}{ZooKeeper简单介绍}
\end{enumerate}

\end{document} 
