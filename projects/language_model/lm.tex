\ifx\projectsnotes\undefined
    \providecommand{\notesroot}{../..}
    \providecommand{\lmroot}{.}

    \title{语言模型}
    \author{Donald Cheung\\jianzhang9102@gmail.com}
    \date{\today\footnote{文档编写开始于2018年05月06日}}

    \documentclass[a4paper,10pt]{ctexbook}
\usepackage{xeCJK}
\usepackage{fontspec}
\usepackage{minted}
\usepackage[CJKbookmarks,colorlinks,linkcolor=red]{hyperref}
\usepackage{geometry}
\usepackage{amsmath}
\usepackage[format=hang,font=small,textfont=it]{caption}
\usepackage{float}
\usepackage{subfigure}
\usepackage[nottoc]{tocbibind}
\usepackage{bm}
\usepackage[table, x11names, dvipsnames]{xcolor}
\usepackage{color}
\usepackage{array, booktabs, boldline}
\usepackage{cellspace}
\usepackage{longtable}

\setmainfont{Times New Roman}
\setsansfont{Helvetica}
\setmonofont{Courier New}
\setCJKmainfont[BoldFont={SimHei},ItalicFont={SimHei}]{SimSun}
\setCJKsansfont{SimSun}
\setCJKmonofont{SimSun}

\setcounter{secnumdepth}{4}
\setcounter{tocdepth}{4}

\geometry{left=3.0cm,right=3.0cm,top=2.5cm,bottom=2.5cm}
\bibliographystyle{plain}

%%%%%%%%%%%%%%%%%%%%%%%%%%%%%%%%% minted setting %%%%%%%%%%%%%%%%%%%%%%%%%%%%%%%%%%%
\usemintedstyle{monokai}
\definecolor{bg}{HTML}{282828} % from https://github.com/kevinsawicki/monokai
%\defaultfontfeatures{}
\newfontfamily\noligsmonofamily[NFSSFamily=noligsmonofamily]{Courier}
\setminted{fontfamily=noligsmonofamily}

\renewcommand{\theFancyVerbLine}{%
    \sffamily \textcolor{Dandelion}{\scriptsize \oldstylenums{\arabic{FancyVerbLine}}}}

\newenvironment{jcode}[3]
{%
    \VerbatimEnvironment
    \begin{listing}[h]%
    \caption{#2}%
    \label{#3}%
    \begin{minted}[xleftmargin=18pt,
                   mathescape,
                   linenos,
                   numbersep=5pt,
                   bgcolor=bg,
                   frame=lines,
                   framesep=2mm,
                   fontsize=\footnotesize]{#1}%
}
{%
    \end{minted}
    \vspace{-25pt}%
    \end{listing}%
}
\renewcommand{\listingscaption}{代码}%from minted
\renewcommand{\listoflistingscaption}{代码列表}% from minted


\newenvironment{myquote}{\begin{quote}\kaishu\zihao{-5}}{\end{quote}}
\newcommand\degree{^\circ}
\newtheorem{thm}{定理}


\begin{document}
\maketitle
\tableofcontents
\listoflistings

\else
    \providecommand{\lmroot}{\projectsroot/language_model}
\fi

\chapter{语言模型}
Language modeling is key to many interesting problems such as speech recognition, machine translation, or image captioning.

\section{数据集介绍}

\subsection{Penn Tree Bank}
宾州树库(Penn Tree Bank)的词性标签及含义如表 \ref{table:ptb_label} 所示。

\begin{longtable}{| p{.05\linewidth} | p{.05\linewidth} | p{.35\linewidth} | p{.30\linewidth} |}
\hline
\rowcolor{SeaGreen3!30!} \textbf{编号} & \textbf{缩写} & \textbf{英文} & \textbf{中文} \\ \hline
1 & CC & Coordinating conjunction & 并列连接词 \\ \hline
2 & CD & Cardinal number & 基数 \\ \hline
3 & DT & Determiner & 限定词 \\ \hline
4 & EX & Existential there & 存在型there \\ \hline
5 & FW & Foreign word & 外文单词 \\ \hline
6 & IN & Preposition/subord, conjunction & 介词/从属,连接词 \\ \hline
7 & JJ & Adjective & 形容词 \\ \hline
8 & JJR & Adjective, comparative & 形容词,比较级 \\ \hline
9 & JJS & Adjective, superlative & 形容词,最高级 \\ \hline
10 & LS & List item marker & 列表项标记 \\ \hline
11 & MD & Modal & 情态动词 \\ \hline
12 & NN & Noun,singular or mass & 名词,可数或不可数 \\ \hline
13 & NNS & Noun, plural & 名词,复数 \\ \hline
14 & NNP & Proper noun, singular & 专有名词,单数 \\ \hline
15 & NNPS & Proper noun, plural & 专有名词,复数 \\ \hline
16 & PDT & Predeterminer & 前位限定词 \\ \hline
17 & POS & Possessive ending & 所有格结束词 \\ \hline
18 & PRP & Personal pronoun & 人称代名词 \\ \hline
19 & PP\$ & Possessive pronoun & 物主代词,所有格代名词 \\ \hline
20 & RB & Adverb & 副词 \\ \hline
21 & RBR & Adverb, comparative & 副词,比较级 \\ \hline
22 & RBS & Adverb, superlative & 副词,最高级 \\ \hline
23 & RP & Particle & 小品词 \\ \hline
24 & SYM & Symbol(mathematical or scientific) & 符号(数学或科学) \\ \hline
25 & TO & to & To \\ \hline
26 & UH & Interjection & 感叹词 \\ \hline
27 & VB & Verb, base form & 动词,基本形态 \\ \hline
28 & VBD & Verb, past tense & 动词,过去式 \\ \hline
29 & VBG & Verb, gerund/present participle & 动词,动名词/现在分词 \\ \hline
30 & VBN & Verb, past participle & 动词,过去分词 \\ \hline
31 & VBP & Verb, non-3rd ps. sing. Present & 动词,非第三人称单数现在式 \\ \hline
32 & VBZ & Verb, 3rd ps. sing. Present & 动词,第三人称单数现在式 \\ \hline
33 & WDT & wh-determiner & wh-限定词 \\ \hline
34 & WP & wh-pronoun & wh-代词 \\ \hline
35 & WP\$ & Possessive wh-pronoun & 所有格wh-代词 \\ \hline
36 & WRB & wh-adverb & wh-副词 \\ \hline
37 & \# & Pound sign & #符号 \\ \hline
38 & \$ & Dollar sign & 美元符号 \\ \hline
39 & . & Sentence-final punctuation & 句点 \\ \hline
40 & , & Comma & 逗号 \\ \hline
41 & : & Colon, semi-colon & 冒号,分号 \\ \hline
42 & ( & Left bracket character & 左括号 \\ \hline
43 & ) & Right bracket character & 右括号 \\ \hline
44 & “ & Straight double quote & 双引号 \\ \hline
45 & ‘ & Left open single quote & 左单引号 \\ \hline
46 & “ & Left open double quote & 左双引号 \\ \hline
47 & ’ & Right close single quote & 右单引号 \\ \hline
48 & ” & Right close double quote & 右双引号 \\ \hline
\caption{PTB标注结果含义}
\label{table:ptb_label}
\centering
\end{longtable}

\subsubsection{相关材料}
\begin{enumerate}
\item \href{https://pdfs.semanticscholar.org/182c/4a4074e8577c7ba5cbbc52249e41270c8d64.pdf}{THE PENN TREEBANK: AN OVERVIEW}
\item \href{https://catalog.ldc.upenn.edu/ldc99t42}{Treebank-3}
\end{enumerate}


\section{RNN语言模型}

\subsection{相关材料}
\begin{enumerate}
\item \href{https://www.tensorflow.org/tutorials/recurrent}{TensorFlow: Recurrent Neural Networks}
\end{enumerate}

\href{http://www.fit.vutbr.cz/~imikolov/rnnlm/simple-examples.tgz}{PTB dataset from Tomas Mikolov's webpage}

The dataset is already preprocessed and contains overall 10000 different words,
including the end-of-sentence marker and a special symbol (<unk>) for rare words.
In reader.py, we convert each word to a unique integer identifier,
in order to make it easy for the neural network to process the data.

The core of the model consists of an LSTM cell that processes one word at a time and
computes probabilities of the possible values for the next word in the sentence.
The memory state of the network is initialized with a vector of zeros
and gets updated after reading each word.
For computational reasons, we will process data in mini-batches of size batch\_size.
In this example, it is important to note that current\_batch\_of\_words does not correspond to a ``sentence" of words.
Every word in a batch should correspond to a time $t$.

\ifx\projectsnotes\undefined
    \bibliography{\notesroot/reference/reference.bib}
\end{document}

\fi
