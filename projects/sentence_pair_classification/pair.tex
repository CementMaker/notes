\ifx\projectsnotes\undefined
    \providecommand{\notesroot}{../..}
    \providecommand{\sentencepairroot}{.}

    \title{句对匹配}
    \author{Donald Cheung\\jianzhang9102@gmail.com}
    \date{\today\footnote{文档编写开始于2018年5月17日}}

    \documentclass[a4paper,10pt]{ctexbook}
\usepackage{xeCJK}
\usepackage{fontspec}
\usepackage{minted}
\usepackage[CJKbookmarks,colorlinks,linkcolor=red]{hyperref}
\usepackage{geometry}
\usepackage{amsmath}
\usepackage[format=hang,font=small,textfont=it]{caption}
\usepackage{float}
\usepackage{subfigure}
\usepackage[nottoc]{tocbibind}
\usepackage{bm}
\usepackage[table, x11names, dvipsnames]{xcolor}
\usepackage{color}
\usepackage{array, booktabs, boldline}
\usepackage{cellspace}
\usepackage{longtable}

\setmainfont{Times New Roman}
\setsansfont{Helvetica}
\setmonofont{Courier New}
\setCJKmainfont[BoldFont={SimHei},ItalicFont={SimHei}]{SimSun}
\setCJKsansfont{SimSun}
\setCJKmonofont{SimSun}

\setcounter{secnumdepth}{4}
\setcounter{tocdepth}{4}

\geometry{left=3.0cm,right=3.0cm,top=2.5cm,bottom=2.5cm}
\bibliographystyle{plain}

%%%%%%%%%%%%%%%%%%%%%%%%%%%%%%%%% minted setting %%%%%%%%%%%%%%%%%%%%%%%%%%%%%%%%%%%
\usemintedstyle{monokai}
\definecolor{bg}{HTML}{282828} % from https://github.com/kevinsawicki/monokai
%\defaultfontfeatures{}
\newfontfamily\noligsmonofamily[NFSSFamily=noligsmonofamily]{Courier}
\setminted{fontfamily=noligsmonofamily}

\renewcommand{\theFancyVerbLine}{%
    \sffamily \textcolor{Dandelion}{\scriptsize \oldstylenums{\arabic{FancyVerbLine}}}}

\newenvironment{jcode}[3]
{%
    \VerbatimEnvironment
    \begin{listing}[h]%
    \caption{#2}%
    \label{#3}%
    \begin{minted}[xleftmargin=18pt,
                   mathescape,
                   linenos,
                   numbersep=5pt,
                   bgcolor=bg,
                   frame=lines,
                   framesep=2mm,
                   fontsize=\footnotesize]{#1}%
}
{%
    \end{minted}
    \vspace{-25pt}%
    \end{listing}%
}
\renewcommand{\listingscaption}{代码}%from minted
\renewcommand{\listoflistingscaption}{代码列表}% from minted


\newenvironment{myquote}{\begin{quote}\kaishu\zihao{-5}}{\end{quote}}
\newcommand\degree{^\circ}
\newtheorem{thm}{定理}


\begin{document}
\maketitle
\tableofcontents
\listoflistings

\else
    \providecommand{\sentencepairroot}{\projectsroot/sentence_pair_classification}
\fi

\chapter{句对匹配}

\section{数据集介绍}

\subsection{Quora Question Pairs}
官方比赛链接:\href{https://www.kaggle.com/c/quora-question-pairs}{Quora Question Pairs}

数据集链接:\url{https://pan.baidu.com/s/1mij3Nza}

\subsubsection{问题详情}
\subsubsection{背景}
Quora是美国版的知乎,月活用户超过1亿,很多人会问很类似的问题。重复问题的出现,会使得信息需求者花费更多的时间去寻找答案,而答主需要对一样的问题解答很多遍。

此数据需要解决的问题就是,判断一对问题是否是重复性问题,也就是说判断两个问题是否是同一个意思。

需要注意的是,数据集中的lable是由人为标注的,也就是说并不是100\%正确的。

\begin{table}[!htb]
    \centering
    \begin{tabular}{|Sc|c|c|c|}
        \hlineB{2}
        \rowcolor{SeaGreen3!30!} \textbf{id} & \textbf{question1} & \textbf{question2} & \textbf{is\_duplicate} \\ \hlineB{1.5}
        1 & Why my answers are collapsed? & Why is my answer collapsed at once? & 0 \\ \hlineB{1.5}
        2 & How do I post a question in Quora? & How do I ask a question in Quora? & 1 \\ \hlineB{1.5}
        3 & Can I fit my booboos in a 65ml jar? & Is 1 baba worth 55 booboo (おっぱい) ? & 0 \\
        \hlineB{2}
    \end{tabular}
    \caption{quora数据集样本示例}
    \label{table:quora_data_sample}
\end{table}

需要注意的是,原始的数据文件以 $\backslash$r$\backslash$n 表示换行符。

\subsubsection{评估}
用log loss作为评估标准。

\section{Quora Question Pairs实验}

\begin{enumerate}
\item 基于词频的cosine相似度计算:[2017-03-21 21:20]
成绩:0.75402

做法:直接使用sklearn的CountVectorizer,然后获取每个问题的切词列表,统计出频次,计算两个向量的相似度。

TODO: 
\begin{itemize}
\item 基于tf-idf特征的方法
\item 两个问题的BOW向量,直接通过LR训练
\item 获取word的Embedding向量,然后加和取平均,获得问题的向量,然后(1)计算两个向量的相似度;(2)直接拼接两个向量,接入LR训练模型
\item 使用论文`Distributed Representations of Sentences and Documents`中的想法
\item 使用fastText工具
\item 使用LSTM
\item 方法参考:https://www.kaggle.com/c/quora-question-pairs/discussion/30340
\end{itemize}



\end{enumerate}

\subsection{参考资料}

\begin{enumerate}
    \item \href{https://github.com/siarez/sentence_pair_classifier}{Tensorflow implementation of Decomposable Attention Model} \\
        这份代码参考了论文 \cite{Parikh:2016aa},并采用TensorFlow实现。

\end{enumerate}


\section{相关参考文献}

\ifx\projectsnotes\undefined
    \bibliography{\notesroot/reference/reference.bib}
\end{document}

\fi
